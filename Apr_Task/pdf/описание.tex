\documentclass{article}
\usepackage[utf8]{inputenc}
\usepackage[russian]{babel}
\usepackage{amsmath}

\begin{document}

\section*{Архитектура клиент-серверной системы для работы с графами}



\subsection*{Общая структура}
Система состоит из двух основных компонентов:
\begin{itemize}
    \item \textbf{Сервер} (на C++): Обрабатывает графы и вычисляет эйлеровы пути
    \item \textbf{Клиент} (на C++): Отправляет запросы и получает результаты
\end{itemize}

\subsection*{Серверная часть}
Реализует следующие алгоритмы:
\begin{enumerate}
    \item Проверка условий существования эйлерова пути:
    \[
    \text{deg}(v) \equiv 
    \begin{cases} 
    \text{чёт} & \forall v \in V \quad \text{(цикл)} \\
    \text{2 нечёт} & \text{(путь)}
    \end{cases}
    \]
    
    \item Алгоритм Флёри для поиска пути:
    \begin{itemize}
        \item Сложность: $O(E^2)$
    \end{itemize}
\end{enumerate}

\subsection*{Клиентская часть}
Функциональность:
\begin{itemize}
    \item Генерация тестовых графов
    \item Валидация входных данных
\end{itemize}

\subsection*{Технические детали}
\begin{itemize}
    \item Язык программирования: C++11
    \item Используемые библиотеки: STL, возможно Boost для работы с графами
    \item Система сборки: CMake. 
    \subsubsection*{Инструкция по сборке и запуску:}
    \begin{itemize}
        \item Скачать репозиторий
        \item Зайти в папку build
        \item Прописать следующие команды: \textbf{cmake ..} а затем \textbf{make}
        \item На одной машине запустить сервер: \textbf{./server}, а на другой клиент: \textbf{./client <IP сервера> <количество потоков (<5)>}
        \item Лучше перед новым запуском удалять файлы \textbf{input} и \textbf{output}, так как при каждом запуске программы в них записываются файлы и не удаляются при новом запуске программы. Эти файлы накапливаются и программа постоянно будет обрабатывать уже обработанные файлы.
    \end{itemize}
    \item Тестирование: включает графы разного размера (только количесвто вершин от 10-50, иначе очень мало тестов будет завершаться успешно, тесты рандомно сгенерированы)
\end{itemize}

\end{document}